\documentclass[10pt,nofootinbib]{article}
\usepackage{epsfig}
\usepackage{graphicx}
\usepackage{color}
\usepackage{amsmath}
\usepackage{amssymb}

\setlength{\textheight}{24cm}
\setlength{\oddsidemargin}{.5cm}
\setlength{\textwidth}{15.4cm}
\setlength{\topmargin}{-2cm}
\setlength{\parindent}{0cm}
\newcommand{\todo}[1]{{\color{blue}$\blacksquare$~\textsf{[TODO:#1]}}}
\newcommand{\code}[1]{\texttt{#1}}

\newenvironment{xaroundtbl}{\begin{flushleft}
  \renewcommand{\footnoterule}{}
  \begin{minipage}{\linewidth}
  \begin{center}}%5
 {\end{center}\end{minipage}\end{flushleft}} 

\title{Finite Temperature EOS driver \& Tablulated EOSs}
\author{C.\ D.\ Ott and E.\ O'Connor}

\begin{document}


\maketitle

\section{EOS Driver}

Here we describe the set of functions used to access the EOS.  Please
also review \code{driver.F90} for usage of the EOS routines.

\subsection{\code{readtable(`pathtoEOS.h5')}}

This function must be called to initialize the EOS and read in the
data from file.

\subsection{\code{add\_index(`name of data set',ivariable)}}

The main EOS table must be read in with \code{readtable}, however,
this only reads in the default 19 variables that must be stored in the
EOS table for compatibility with \code{EOSdriver}.  There may be
additional variables in the .h5 file that the user may wish to access.
A call to \code{add\_index} with the appropiate HDF5 dataset name will
add this EOS variable to the main EOS array stored in memory.  The
index of this variable is return in \code{ivariable}.  Use this with
\code{nuc\_eos\_one} (described below) to interpolate this additional
variable.

\subsection{\code{nuc\_eos\_full}}
\subsection{\code{nuc\_eos\_short}}
\subsection{\code{nuc\_eos\_one}}

This routine interpolated one and only one EOS variable.
Interpolating one variable at a time is inefficient as there is a lot
of overhead, however this may be acceptable in some situations.  To
use \code{nuc\_eos\_one}, provide as arguements \code{rho},
\code{temp}, \code{ye}, \code{interpolated\_value}, and
\code{index\_of\_variable}, respectively. The last
arguement is either returned by \code{add\_index} when the variable is
added to the table, or can be found in \code{eosmodule.F90}. This
routine does not accept the internal energy, entropy, or pressure as
EOS inputs, any EOS inversion must be done separately.



\section{Tabulated EOSs}

\subsection{Table format}

Our EOS driver accepts tablulated EOS in \code{HDF5} format. Table
\ref{tab:eosvariables} lists the required fields and a short
description.  


\begin{table}[ht]
\begin{xaroundtbl}
\begin{tabular}{lll}
Variable & Units & Description \\[0.2mm] \hline
\code{pointsrho} & dimensionless & number of table points in
$\log_{10}(\rho)$ \\[0.2mm]
\code{pointstemp} & dimensionless & number of table points in
$\log_{10}(\rm{T})$ \\[0.2mm]
\code{pointsye} & dimensionless & number of table points in $Y_e$
\\[0.2mm]
\code{logrho} & $\log_{10}(\rho [\rm{g}/\rm{cm}^3])$ & index variable
$\rho$\\[0.2mm]
\code{logrho} & $\log_{10}(T [\rm{MeV}])$ & index variable
T\\[0.2mm]
\code{logrho} & number fraction & index variable
$Y_e$\\[0.2mm]
\code{Abar} & A & average heavy nucleus mass number\\[0.2mm]
\code{Zbar} & Z & average heavy nucleus atomic number\\[0.2mm]
\code{Xa} & mass fraction & $\alpha$ particle mass fraction\\[0.2mm]
\code{Xh} & mass fraction & average heavy nucleus mass
fraction\\[0.2mm]
\code{Xn} & mass fraction & neutron mass fraction\\[0.2mm]
\code{Xp} & mass fraction & proton mass fraction\\[0.2mm]
\code{cs2} & $\rm{cm}^2/\rm{s}^2$ & speed of sound squared\\[0.2mm]
\code{dedt} & $\rm{erg}/\rm{g}/\rm{MeV}$ & $C_v$\\[0.2mm]
\code{dpderho} & $\rm{dynes}\ \rm{g}/\rm{cm}^2/\rm{erg}$ & dP/d$\epsilon$ at constant
$\rho$\\[0.2mm]
\code{dpdrhoe} & $\rm{dynes}\ \rm{cm}^3/\rm{cm}^2/\rm{g}$ & dP/d$\rho$ at constant
$\epsilon$\\[0.2mm]
\code{energy\_shift} & $\rm{erg}/\rm{g}$ & energy shift for table storage
\footnote{see below}\\[0.2mm]
\code{entropy} & $k_B$/baryon & specific entropy\\[0.2mm]
\code{gamma} & dimensionless & d$\log{[P]}$/d$\log{[\rho]}$\\[0.2mm]
\code{logenergy} & $\log_{10}(\epsilon [\rm{erg}/\rm{g}])$ & specific internal
energy\\[0.2mm]
\code{logpress} & $\log_{10}(P [\rm{dynes}/\rm{cm}^2])$ & pressure\\[0.2mm]
\code{mu\_e} & MeV/baryon & electron chemical
potential\footnote{includes rest mass}\\[0.2mm]
\code{mu\_p} & MeV/baryon & proton chemical potential\footnote{includes rest mass, see specific EOS for detials}\\[0.2mm]
\code{mu\_n} & MeV/baryon & neutron chemical potential\footnote{includes rest mass, see specific EOS for detials}\\[0.2mm]
\code{muhat} & Mev/baryon & mu\_n - mu\_p\\[0.2mm]
\code{munu} & Mev/baryon & mu\_e - muhat
\end{tabular}
\end{xaroundtbl}
\caption{EOS driver \code{HDF5} variables}\label{tab:eosvariables}
\end{table} 



\subsection{Shen EOS}

\subsubsection{Table Construction}                

Our Shen EOS is constructed on the basis of the Shen et al. 1998
relativistic-mean field nuclear EOS table. Electrons (fully general,
based on TimmesEOS) and Photons are added.
\newline

Original Shen EOS table extent:
\begin{table}[ht]
\begin{tabular}{r|l}
Density & $10^{5.1}$ - $10^{15.4}\ \rm{g}/\rm{cm}^3$ \\[0.2mm]
Temperature & $0.1$ - $100$ MeV \\[0.2mm]
$Y_e$ & $0.01$ - $0.56$
\end{tabular}
\end{table}

Table extent of current table [\code{myshen\_test\_220r\_180t\_50y\_extT\_20090312.h5}]:
\begin{table}[ht]
\begin{tabular}{r|l}
Density & $10^{3}$ - $10^{15.36}\ \rm{g}/\rm{cm}^3$ \\[0.2mm]
Temperature & $0.01$ - $250$ MeV \\[0.2mm]
$Y_e$ & $0.015$ - $0.56$
\end{tabular}
\end{table}

This bigger table is realized by extending the original Shen table in
multiple ways in multiple directions:\newline
\newline
 
(a) density: \newline Match of pure ideal gas of Ni$^{56}$ +
electrons/positrons + photons at densities below $10^7
\rm{g}/\rm{cm}^3$ -- at this density pressures, energies and entropies
match okayish with the values in the Shen table. The compositions
(\code{Abar},\code{Zbar},\code{Xh},\code{Xa},\code{Xp},\code{Xn}) are
kept constant in the low-density region and \code{mu\_n} and
\code{mu\_p} are set to 0 -- ideally, at low densities, a full NSE EOS
with nuclear reaction network (at low $T$) should be stitched onto the
Shen; working on that, but not yet ready.  \newline \newline

(b) temperature (extrapolation): \newline At high density: linear
extrapolation of everything in $T$ to lower temperatures and higher
temperatures. At low densities (below $10^7 \rm{g}/\rm{cm}^3$), ideal
gas of Ni$^{56}$ + electrons/positrons + photons.

\subsubsection{Chemical Potentials}
The nucleon chemical potentials are fully relativistic in the Shen
EOS.  They include the rest mass but are given with respect to a
mass of $M = 938$~MeV, i.e. $\mu_n = \tilde{\mu}_n - M$.  Therefore
$\hat{\mu} = \mu_n-\mu_p$ includes the neutron-proton mass difference.

\subsubsection{Energy Shift}
In some regions the negative nuclear binding energy is larger in
magnitude than the thermal/excitation energy. In this case the
specific internal energy ($\epsilon$) becomes negative. To allow for storage
and interpolation of $\epsilon$ in logarithmic fashion, the energy is shifted
up by an energy shift specified in the variable \code{energy\_shift}. This
energy shift is handled internally in the EOS routines.

\subsection{LS EOSs}

\subsubsection{Chemical Potentials}
The nucleon chemical potentials are fully relativistic in the LS EOSs
 in the sense that they include the rest mass of the particles.  The
 chemical potentials are given with respect to the neutron rest mass.
 Therefore $\hat{\mu} = \mu_n-\mu_p$ includes the neutron-proton mass
 difference.

\subsubsection{Energy Shift}
In some regions the negative nuclear binding energy is larger in
magnitude than the thermal/excitation energy. In this case the
specific internal energy ($\epsilon$) becomes negative. To allow for storage
and interpolation of $\epsilon$ in logarithmic fashion, the energy is shifted
up by an energy shift specified in the variable \code{energy\_shift}. This
energy shift is handled internally in the EOS routines.

\end{document}    
